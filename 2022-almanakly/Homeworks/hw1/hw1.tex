\documentclass[11pt]{article}
\usepackage{amsmath,amssymb,graphicx,enumerate}
\usepackage{hyperref}
\usepackage[parfill]{parskip}
\hypersetup{
    colorlinks=true,
    linkcolor=blue,
    filecolor=magenta,      
    urlcolor=blue,
}

\def\Homework{1} % Number of Homework
\def\Session{Spring 2022}
\def\Section{B}
\def\MyEmail{husam.almanakly@cooper.edu}
\def\DateOfSubmission{January 26, 2022 }

\title{MATLAB Assignment \Homework}
\author{\Session, Section \Section}
\date{}

\newenvironment{qparts}{\begin{enumerate}[{(}a{)}]}{\end{enumerate}}

\textheight=9in
\textwidth=6.5in
\topmargin=-.75in
\oddsidemargin=0.25in
\evensidemargin=0.25in


\begin{document}
\maketitle

This homework is designed to teach you to think in terms of matrices and vectors because this is how MATLAB organizes its data. You will find that complicated operations can be resolved by one to two lines of code if you use the correct function and store data in the appropriate form. The other purpose of this homework is to make you feel comfortable with using \textbf{\textit{help}} function. 

It's a good habit to begin all your scripts with clc, clear, and close all. Remember to practice good coding style! Keep variable names consistent, use comments to explain code where needed, use sections and spacing consistently etc. 

For the final submission, please suppress all outputs using semicolons. Submit your *.m file by the end of the day on \DateOfSubmission in your Teams private channel. 


\noindent \textbf{1. Creating Scalar Variables} Create the following variables. Each construction should be done in \textbf{one} line. Make sure to use the assigned variable names. 
\begin{qparts}
\item
$ a = \frac{5.7 \pi}{6.9} $
\item 
$ b = 239+e^5 - 2.5 \times 10^{23}$
\item
$ c = ln(4.23) \times sin^{-1}(0.7)$
\item 
$ z = (3+2j) \times (4+5j) $

\end{qparts}

\noindent \textbf{2. Complex Operations} Find the real part, imaginary part, magnitude, phase and complex conjugate of $z$ calculated in question 1e.

\noindent \textbf{3. Vector and Matrix Variables} Create the following variables. Make sure to use the assigned variable names. When doing part c and d, make sure you know when to use the colon operator $:$ , and when to use \textbf{\textit{linspace}}. 

\begin{qparts}
\item
Create a row vector where $ aVec = \begin{bmatrix}3.14&15&9&26+0.1j\end{bmatrix}$, and generate matrices $A1$ and $A2$ with \textbf{\textit{repmat}} and concatenation respectively, where 

$A1 = A2 = \begin{bmatrix}3.14&15&9& 26+0.1j \\ 3.14&15&9&26+0.1j\\3.14&15&9&26+0.1j\end{bmatrix}$ 
\item
Create the column version of $aVec$, with both matrix constructor operation $[$ $]$ and \textbf{\textit{transpose}} function in MATLAB. Name the variables $bVec1$ and $bVec2$ respectively.    
\item  
Create a row Vector $cVec$ where the numbers ranges from -5 to 5 in increasing order and at an interval of 0.1 between consecutive numbers. 
\item
Create a column vector $dVec$ where there are 100 evenly spaced points between -5 and 5. Do not use the same operator in part c. \textbf{Optional} : Can you do it in one line?
\item
Create a matrix $B$ where $B = \begin{bmatrix} 1+2j&10^{-5}\\ e^{j2\pi}&3+4j \end{bmatrix}$
\item 
Use \textbf{\textit{eye}} to ATTEMPT to create a $1,000,000 \times 1,000,000$ identity matrix.
\item
Use \textbf{\textit{speye}} to create a $1,000,000 \times 1,000,000$ sparse identity matrix. (Suppress the output with a semicolon)
\end{qparts}

\noindent \textbf{4. Vector and Matrix Operations} Using the variables made in question 3, perform the following operations:
\begin{qparts}
\item
Use \textbf{\textit{magic}} and divide by 65 to create a $5 \times 5$ doubly stochastic matrix $A$.
\item
Create a $5 \times 5$ matrix, $B$, such that each element is drawn from standard normal distribution. (Note: You'll need to look up how to make it).
\item
Compute $C = BA$.
\item
Compute $D = BA$, but different from part c, perform element wise multiplication.
\item
Compute $F = \frac{1}{4}A^3 + \frac{1}{4}A^2 + \frac{1}{3}A + \frac{1}{6}I$. 
\item
Compute $G = A^{-1}$.
\end{qparts}

\end{document}

