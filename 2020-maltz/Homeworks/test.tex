
% This LaTeX was auto-generated from MATLAB code.
% To make changes, update the MATLAB code and republish this document.

\documentclass{article}
\usepackage{graphicx}
\usepackage{color}

\sloppy
\definecolor{lightgray}{gray}{0.5}
\setlength{\parindent}{0pt}

\begin{document}

    
    
\section*{ECE302 Project 2: Estimation Techniques}


\subsection*{Contents}

\begin{itemize}
\setlength{\itemsep}{-1ex}
   \item Scenario 1
   \item Bayes MMSE Estimator
\end{itemize}


\subsection*{Scenario 1}

\begin{par}
An implementation of a Bayes MMSE and Linear MMSE estimators of the random variable Y from the random variable X where X = Y + W. Here Y \ensuremath{\tilde{\;}} U(-1,1) and W \ensuremath{\tilde{\;}} U(-2,2). Note that in this project, N represents the number of samples taken from the respective distributions.
\end{par} \vspace{1em}
\begin{verbatim}
clear;
clc;
close all;

N = 10000;

Y1 = random('Uniform',-1,1,N,1);
W = random('Uniform',-2,2,N,1);
X1 = Y1 + W;
\end{verbatim}


\subsection*{Bayes MMSE Estimator}

\begin{par}
Here we take $\hat{y} = E[Y | X = x]$. Calculating this condition expectation leads to: \begin{verbatim}latex\end{verbatim} \ensuremath{\backslash}usepackage\{mathworks\} \ensuremath{\backslash}[ \ensuremath{\backslash}hat\{y\} =       \ensuremath{\backslash}begin\{cases\}       0 \& x\ensuremath{\backslash}leq 0 \ensuremath{\backslash}\ensuremath{\backslash}       \ensuremath{\backslash}frac\{100-x\}\{100\} \& 0\ensuremath{\backslash}leq x\ensuremath{\backslash}leq 100 \ensuremath{\backslash}\ensuremath{\backslash}       0 \& 100\ensuremath{\backslash}leq x    \ensuremath{\backslash}end\{cases\} \ensuremath{\backslash}] \begin{verbatim}/latex\end{verbatim}
\end{par} \vspace{1em}
\begin{verbatim}
Y1est = zeros(N,1);
for i = 1:N
    if X1(i) < -1
        Y1est(i) = .5 + .5*X1(i);
    elseif X1(i) < 1
        Y1est(i) = 0;
    else
        Y1est(i) = -.5 + .5*X1(i);
    end
end

empmmse1 = mean((Y1 - Y1est).^2);

actmmse1 = .25;
\end{verbatim}
\begin{verbatim}
Y1est = (1/5)*X1;

emplmmse1 = mean((Y1 - Y1est).^2);

actlmmse1 = 4/15;
\end{verbatim}
\begin{verbatim}
table([empmmse1; emplmmsel],[actmmse1; actlmmse1],'VariableNames', ...
    ["Empirical MSE","Theorhetical MSE"],'RowNames', ...
    ["MMSE","LMMSE"])
\end{verbatim}

        \color{lightgray} \begin{verbatim}Unrecognized function or variable 'emplmmsel'.

Error in proj2 (line 55)
table([empmmse1; emplmmsel],[actmmse1; actlmmse1],'VariableNames', ...
\end{verbatim} \color{black}
    \begin{verbatim}
m = 5;
n = 5;

Y2est = zeros(N,m);
emplmmse2 = zeros(1,m);
actlmmse2 = zeros(1,m);
leg = strings(1,m);

for j = 1:m
    muY2 = 1;
    varY2 = j;
    varR = j;

    Y2 = random('Normal',muY2,sqrt(varY2),N,1);
    R = random('Normal',muR,sqrt(varR),N,n);
    X2 = Y2 + R;

    CXX = varY2*ones(n) + diag(varR*ones(1,n));
    CXY = varY2*ones(n,1);

    a = CXX\CXY;

    a0 = muY2 - dot(a,muY2*ones(n,1));

    Y2est(:,j) = a0 + dot(repmat(a',N,1),X2,2);

    emplmmse2(j) = mean((Y2 - Y2est(:,j)).^2);

    actlmmse2(j) = varY2 - CXY'*a;

    leg(j) = "\sigma_Y^2 = " + j + ", \sigma_R^2 = " + j;
end

sz = 25;

figure;
for j = 1:m
    scatter(emplmmse2(j),actlmmse2(j),sz,j,'filled','DisplayName',leg(j));
    hold on;
end
xlabel("Empirical LMMSE");
ylabel("Theorhetical LMMSE");
legend('location','northwest');
grid on;
title("LMMSE with " + n + " observations");
\end{verbatim}



\end{document}
    
