\documentclass[12pt]{article}
\usepackage{geometry}
\title{ECE-210-B Homework 8}
\author{Samuel Maltz}
\date{Spring 2020}
\usepackage{fancyhdr}
\usepackage{amsmath}

\pagestyle{fancy}
\fancyhf{}
\rhead{Samuel Maltz}
\lhead{ECE-210B Homework 8}
\rfoot{\thepage}

\renewcommand{\headrulewidth}{0pt}

\begin{document}
\noindent In this homework, you will explore the topics of the past two lectures: state space and the symbolic toolbox.
\begin{enumerate}
\item Here you will check the functionality of \textbf{\textit{ss2tf}} against the symbolic toolbox through this continuous state space system:
$$A = \begin{bmatrix}
-4 & 4 \\
-2 & 0\end{bmatrix}$$
$$B = \begin{bmatrix}
0 \\
2
\end{bmatrix}$$
$$C = \begin{bmatrix}
1 & 0
\end{bmatrix}$$
$$D = \begin{bmatrix}
0
\end{bmatrix}$$
\begin{itemize}
\item First check (show me) to make sure the real part of all of the eigenvalues are negative, thus making the system stable.
\item Next check the norm of $e^{At}$ over a large enough time $t$ and show in a plot that it decays to 0. Caution: the regular \textbf{\textit{exp}} function is not appropriate for matrices, find a function which is.
\item Now use \textbf{\textit{ss2tf}} and return the transfer function numerator and denominator.
\item Check your work by using the formula $H(s) = C(sI-A)^{-1}B+D$ with a symbolic variable $s$.
\item Extract the numerator and denominator of the symbolic rational function using a single function. Then convert these to polynomial vectors using another single function.
\end{itemize}

\item Consider the system of differential equations below:
$$\frac{dx}{dt} = y$$
$$\frac{dy}{dt} = -x-y$$
Solve the system using the symbolic toolbox's \textbf{\textit{dsolve}} function using four \textit{different} initial conditions (so you should have four sets of solutions):
$$x(0)=1,y(0)=1$$
$$x(0)=-1,y(0)=1$$
$$x(0)=1,y(0)=-1$$
$$x(0)=-1,y(0)=-1$$
Then plot all four curves on the same plot from $t=0$ to $t=100$ using \textbf{\textit{fplot}} (use a legend). It should be a spiral.

\item
\begin{enumerate}
\item Create a function which takes in a matrix $A$ and plots (use \textbf{\textit{semilogy}}) $||e^{At}||_F$ and $e^{\alpha t}$ on the same plot for $0\leq t \leq 100$ (you can sample $t$ at integer values) where $\alpha = max(Re(\Lambda(A)))$ (maximum real part of the eigenvalues of $A$) and $||\cdot||_F$ is the Frobenius norm. In your main script call this function on a random 10x10 matrix with entries sampled from the standard normal distribution minus 2 times the identity. Check out what happens when $\alpha$ is positive or negative and check out what happens when the eigenvalue which $\alpha$ corresponds to is complex and write what happens in a comment.

\item Create a function which takes in a matrix $A$ and plots (use \textbf{\textit{semilogy}}) $||A^n||_F$ and $\rho^n$ on the same plot for $0\leq n \leq 100$ (you can sample $n$ at integer values) where $\rho = max(|\Lambda(A)|)$ (maximum magnitude of the eigenvalues of $A$) and $||\cdot||_F$ is the Frobenius norm. In your main script call this function on a random 10x10 matrix with entries sampled from a normal distribution with mean 0, variance 0.1. Check out what happens when $\rho$ is less than or greater than 1 and check out what happens when the eigenvalue which $\rho$ corresponds to is complex and write what happens in a comment.
\end{enumerate}
\end{enumerate}
\end{document}