\documentclass[12pt]{article}
\usepackage{geometry}
\title{ECE-210-B Homework 6}
\author{Samuel Maltz}
\date{Spring 2020}
\usepackage{fancyhdr}

\pagestyle{fancy}
\fancyhf{}
\rhead{Samuel Maltz}
\lhead{ECE-210B Homework 6}
\rfoot{\thepage}

\renewcommand{\headrulewidth}{0pt}

\begin{document}
\noindent This homework deals with digital filters in a low-level sense. You are expected to know a bit about the z-transform, but if you are not in Signals and Systems, please contact me separately for some additional information on this homework if you need it. Read the parts carefully and make sure to complete every part of the homework. This homework requires you to produce a few plots; I want nice plots! Axis labels and titles are a must.

\begin{enumerate}
\item For this question, you will be working with the discrete system described by the transfer function:
$$H(z) = \frac{\frac{2}{5}z^2 + \frac{1}{4}z + \frac{1}{7}}{\frac{1}{3}z^3 - \frac{1}{8}z + \frac{3}{2}}$$
	\begin{itemize}
	\item Store this transfer function as numerator and denominator polynomials. Be \textbf{VERY} careful setting this up. Check the documentation for \textbf{\textit{zplane}} to see how discrete transfer functions are handled in MATLAB.
	\item Compute the poles and zeros using a specific MATLAB function. (Make sure you use the right one for discrete signals and not the one used mostly for continuous signals!)
	\item Create a poly zero plot using a different MATLAB function. You may use either the poles and zeros themselves or the numerator and denominator polynomials.
	\item Use \textbf{\textit{impz}} to compute the impulse response of this transfer function. Compute only the first 50 points of it (there is a way to do this in the function itself). Make a stem plot of the impulse response.
	\item Let $x[n] = (-\frac{4}{5})^n$. Then use \textbf{\textit{filter}} to apply the transfer function (or filter) to the signal. In subplots, plot the before and after.
	\item  The above is the easiest way to apply a filter, but you also ought to be able to do this analytically, using either convolution in MATLAB or the product of z-transforms. Show me you know how to do this! That is, plot the same answer achieved in another way. You don't have to take any inverse z-transforms to do this! Note that if you use convolution with the impulse response, you'll get a longer vector than when you used filter. Therefore, only plot the first n points where n is the length of the vector result from the previous part.
	\end{itemize}
\item In this question, you will be "designing" a bandpass filter (probably unrealistic but it's what I came up with). A bandpass filter is a system which only allows a certain band of frequencies from a signal to pass.
\begin{itemize}
	\item Last question you converted a transfer function in tf form to one in zpk form. Now we will do the opposite. Compute numerator and denominator vectors for a transfer function with $k = 0.01$ and these zeros and poles:
	$$zeros: -1,\: 1$$
	$$poles: 0.9e^{j\frac{\pi}{2}},\: 0.9e^{-j\frac{\pi}{2}},\: 0.95e^{j\frac{5\pi}{12}},\: 0.95e^{-j\frac{5\pi}{12}},\: 0.95e^{j\frac{7\pi}{12}},\: 0.95e^{-j\frac{7\pi}{12}}$$
	\item Create a pole-zero plot.
	\item Now compute the frequency response of this filter using \textbf{\textit{freqz}}. Use $n = 1024$ points and return an $H$ frequency response vector and a $w$ frequency vector. \textbf{DO NOT} use the option to plot the frequency response. You will be manually creating the plots.
	\item The $H$ vector is a complex vector. That means it has both magnitude gain (via \textbf{\textit{abs}}) and phase (via \textbf{\textit{angle}}). Using subplots, plot the magnitude and phase against the $w$ frequency vector. A few things to note (and will be expected in addition to proper plots):
	\begin{itemize}
	\item Plot the magnitude in dB. You can convert a gain $x$ to dB via $20log_{10}(x)$.
	\item Plot the phase in degrees. You will notice if you do that, the plot will have weird sharp edges. Remove them using \textbf{\textit{unwrap}} \textit{before} converting to degrees.
	\item Show units in the axis labels. (Remember the frequency vector, $w$, has units \textit{radians} not radians per second.)
	\item Remember the frequency vector, $w$, only goes from 0 to $\pi$. Make sure to use \textbf{\textit{xlim}}, \textbf{\textit{xticks}} and \textbf{\textit{xticklabels}}.
	\end{itemize}
\end{itemize}
\end{enumerate}
\end{document}