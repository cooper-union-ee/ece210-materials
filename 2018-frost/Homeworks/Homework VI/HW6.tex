\documentclass{article}
\usepackage{geometry}
\usepackage{amsfonts}
\usepackage{amsmath}
\usepackage{enumitem}
\usepackage{titling}
\title{ECE-210-A Assignment VI}
\author{Brian Frost-LaPlante}
\date{Spring 2018}

\pretitle{\begin{center}\huge}
\posttitle{\end{center}}
\preauthor{\begin{center}\small}
\postauthor{\end{center}}
\predate{\begin{center}\footnotesize}
\postdate{\end{center}}
\setlength{\droptitle}{-40pt}

\begin{document}
\maketitle
\noindent This homework deals with digital filters in a low-level sense. You are expected to know a bit about the z-transform, but if you are not in Signals and Systems, please contact me separately for some additional information on this homework. This homework requires you to produce a few plots; I want nice plots! Axis labels and titles are a must.
\\\\
\begin{enumerate}[leftmargin=0cm,itemindent=.5cm,labelwidth=\itemindent,labelsep=0cm,align=left,label=\textbf{\arabic*.}]
\item For this question, you will be working with the discrete system described by the transfer function 
$$H(z) = \frac{\frac{1}{2}+\frac{2}{3}z+\frac{3}{7}z^2}{2+\frac{1}{3}z+\frac{1}{2}z^3}$$
\begin{itemize}
	\item $\:\:$ Store this transfer function in MATLAB as numerator and denominator vectors, then use \textit{single} built-in functions to find the poles and zeros and plot the poles and zeros of $H$.
	\item $\:\:$ Use \textit{\textbf{impz}} to obtain the first 50 points of the impulse response and plot them using the appropriate plotting function. 
	\item $\:\:$ Let $x[n] = (-\frac{3}{4})^n$, and take n from 0 to 99. Apply the digital filter $H$ to $x$ using \textit{\textbf{filter}} and plot the result. 
	\item $\:\:$ The above is the easiest way to apply a filter, but you also ought to be able to do this analytically, using either convolution in MATLAB or the product of z-transforms. Show me you know how to do this! That is, plot the same answer achieved in another way. You don't have to take any inverse z-transforms to do this!
\end{itemize}
\item The Fibonacci sequence is the sequence of numbers such that every number after the first two is the sum of the two preceding numbers. The first two numbers are both 1. It is cute to imagine a discrete-tyme system whose impulse response is the Fibonacci sequence! Use a for loop to generate the first 100 values of the Fibonacci sequence and plot these values using MATLAB and plot them using \textit{\textbf{semilogy}}. Assuming this is the impulse response of a system, find the output of the system with input $x[n]$ from the above problem and plot it using an appropriate function.
\end{enumerate}
\end{document}