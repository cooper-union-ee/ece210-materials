\documentclass[11pt]{article}
\usepackage{amsmath,amssymb,graphicx,enumerate}
\usepackage{hyperref}
\usepackage{titling}
\usepackage[parfill]{parskip}
\hypersetup{
    colorlinks=true,
    linkcolor=blue,
    filecolor=magenta,      
    urlcolor=blue,
}

\title{ECE-210-A Assignment VII}
\author{Brian Frost-LaPlante}
\date{Spring 2018}

\pretitle{\begin{center}\huge}
\posttitle{\end{center}}
\preauthor{\begin{center}\small}
\postauthor{\end{center}}
\predate{\begin{center}\footnotesize}
\postdate{\end{center}}
\setlength{\droptitle}{-40pt}

\newenvironment{qparts}{\begin{enumerate}[{(}a{)}]}{\end{enumerate}}

\textheight=9in
\textwidth=6.5in
\topmargin=-.75in
\oddsidemargin=0.25in
\evensidemargin=0.25in


\begin{document}
\maketitle
In this assignment, you will reinforce what we did in lecture today regarding MATLAB's filter toolbox. Please include all your work in a .m file. This assignment is due in two weeks, and has been assigned in all three sections.\\

For each of the following questions, generate filters using either \emph{fdatool} or the filter design toolbox in the signal processing toolbox. Apply the filter to the signal using \emph{filter} . Lastly, plot the Fourier Transform of the final result using \emph{fft} and \emph{plot}. Refer to the notes for the proper way to use \emph{fft} and obtain the proper scaling

1. Generate a signal that consists of a sum of sine waves of frequencies 1 to 50 kHz. Set t to be from 0 to 2 seconds, using an interval of 0.001s. 

$$ signal = (\sum_{f=1}^{50000} sin(2\pi ft))$$

2. Create a Butterworth lowpass filter with a sampling frequency of Fs = 100 kHz, a passband frequency of Fpass = 10 kHz, a stopband frequency of Fstop = 20 kHz, a passband attenuation of Apass = 5dB, and a stopband attenuation of Astop = 50dB.

3. Create a Chebychev I highpass filter with a sampling frequency of Fs = 100 kHz, a passband frequency of Fpass = 35 kHz, a stopband frequency of Fstop = 15 kHz, a passband attenuation of Apass = 2dB, and a stopband attenuation of Astop = 40dB.

4. Create a Chebychev II bandstop filter with a sampling frequency of Fs = 100 kHz, a passband frequency of below the frequency Fpass1 = 5 kHz and above Fpass2 = 45 kHz, a stopband frequency of between Fstop1 = 15 kHz Fstop2 = 35kHz, a passband attenuation of Apass = 5dB, and a stopband attenuation of Astop = 50dB.

4. Create a Elliptic bandpass filter with a sampling frequency of Fs = 100 kHz, a stopband frequency of below the frequency Fstop1 = 15 kHz and above Fstop2 = 35 kHz, a passband frequency of between Fpass1 = 20 kHz Fpass2 = 30 kHz, a passband attenuation of Apass = 5dB, and a stopband attenuation of Astop = 50dB.

\end{document}