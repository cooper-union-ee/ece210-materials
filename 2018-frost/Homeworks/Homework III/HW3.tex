\documentclass{article}
\usepackage{geometry}
\usepackage{amsfonts}
\usepackage{amsmath}
\usepackage{enumitem}
\usepackage{titling}
\title{ECE-210-A Assignment III}
\author{Brian Frost-LaPlante}
\date{Spring 2018}

\pretitle{\begin{center}\huge}
\posttitle{\end{center}}
\preauthor{\begin{center}\small}
\postauthor{\end{center}}
\predate{\begin{center}\footnotesize}
\postdate{\end{center}}
\setlength{\droptitle}{-40pt}

\begin{document}
\maketitle
\noindent This homework delves deeper into relational and logical indexing of matrices and plotting in MATLAB. We will walk you through three problems that will serve as useful examples of logical indexing. This assignment matches that given in sections B and C.
\\\\
\noindent \textbf{1. Lunar Eclipse} This question guides you through some basic image processing techniques in MATLAB. You will create interesting images with relational and logical indexing, as well as the \emph{imshow} function to visualize what you have created.  
\begin{itemize}
\item Create a $100 \times 100$ $A$ where its contents are all ones.
\item Create a $100 \times 100$ $B$ where its contents are all zeros.
\item In matrix $A$, set the values of entry $a_{i,j}$ equal to 0 if $\sqrt{(i-50)^2 + (j-50)^2} < 20$. \textbf{Hint} : \emph{meshgrid} would be useful in creating the indices.
\item In matrix $B$, set the values of entry $a_{i,j}$ equal to 1 if $\sqrt{(i-40)^2 + (j-40)^2} < 20$.
\item Visualize the following results with \emph{figure} and \emph{imshow}. Describe each of the results with one sentence each. 
\begin{itemize}
\item $A$
\item $B$
\item Intersection between $A$ and $B$
\item Union between $A$ and $B$
\item Complement of intersection between $A$ and $B$
\item Complement of union between $A$ and $B$
\end{itemize}
\end{itemize}

\noindent \textbf{2. Fun with find} Write a function to return the value and index of a number in a vector / matrix that is closest to a desired value. The function should be called as $[val, ind] = findClosest(x, desiredValue)$. This function can be accomplished in less than five lines. You will find \emph{abs}, \emph{min} and/or \emph{find} useful, \textbf{Hint}: You may have some trouble using \emph{min} when x is a matrix. To convert the matrix to a vector, you can use $y = x(:)$.  Show that it works by finding the value closest to $3/2$ (and index of said value) in \emph{sin( linspace(0,5,100) ) + 1}.
\\\\
\noindent \textbf{3. Sincing Ship} Here we will look at a function near and dear to the signal processing community's heart - the cardinal sine function, or sinc.  You are going to implement your own functions to find the local extremum and roots.
\begin{itemize}
\item Sample a sinc with 10001 linearly spaced points on $[-2\pi, 2\pi]$ using the \emph{sinc} function.  Create a plot using \emph{plot(x,y)}
\item Create a function (either anonymously or in another file) which locates the indices at which the input vector transitions from one sign to another. \textbf{Note:} This can be done in one line of code but it is $trecherous^{FF}$.  For one scenario the vector has a positive value and then a negative value, i.e. $v(n) > 0$ and $v(n+1) < 0$.  The root occurs somewhere in between, you can pick either $n$ or $n+1$.  We could loop through and check this condition at every point - don't do that.  Instead think of a way to use logical indexing: You will want to write conditions on the vector and some kind of shifted version of itself.  Beware however, when you do this you will have non-overlapping points.  It is up to you to figure out what to with them.
\item Apply your function to the sinc you created.  Find the roots (x and y coordinates) and plot them as black circles on top of the sinc using \emph{plot(xRoots,yRoots,'ko')}.  (make sure your axis is tight.)  
\item Now we are interested in finding the extremum. (local minimums and maximums.) Apply your function to the approximate derivative of your sinc to obtain them, then plot them as red stars on top of the sinc using \emph{plot(xMinMax,yMinMax,'r*')}.
\end{itemize}
\end{document}