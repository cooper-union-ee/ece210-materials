\documentclass{article}

\usepackage{amsfonts}
\usepackage{amsmath}
\usepackage{xcolor}
\usepackage{textcomp}
\usepackage{graphicx}
\usepackage{hyperref}
\usepackage[]{matlab-prettifier}
\usepackage{multirow}

\definecolor{darkblue}{HTML}{2222cc}

\usepackage{listings}
\lstset{
	basicstyle=\ttfamily\color{darkblue}
}

%\begin{lstlisting}[style=Matlab-editor]

\title{ECE-210-B HW7}
\author{Instructor: Jonathan Lam}
\date{Spring 2021}

\begin{document}
	\maketitle
	
	\noindent For each of the following scenarios, you will generate a filter, create magnitude-phase plots for the filter, and apply the filter. First, we will generate a test vector $x$ on which to apply the filters. Let $x$ be the sum of sinusoids of all integer frequencies from 1Hz to 5kHz. Set $t=[0,2]$, using an sampling frequency of 100kHz. I.e.,
	\begin{equation*}
		x=\sum_{f=1}^{50000}\sin(2\pi ft),\ \{0\le t\le 2\}
	\end{equation*}
	For each filter, follow these steps (you may want to define this in a function so you don't have to repeat this code). Subplots recommended!
	\begin{enumerate}
		\item Use the given specifications to produce the lowest-order filter which meets the specs. Either:
		\begin{enumerate}
			\item Use \lstinline|filterDesigner| to generate a MATLAB function that returns a filter, and then call the function to create the filter object; or
			\item Use the functions for designing and estimating the order of specific types of filters (e.g., \lstinline|cheby2ord|, \lstinline|cheby2|) to generate it without \lstinline|filterDesigner|. Make sure all the parameters are correctly specified! Refer to the lecture example.
		\end{enumerate}
		
		\item Apply \lstinline|freqz| on the filter object to produce a frequency-response plot. Similarly to in HW6, don't use this to plot the frequency response: manually plot the frequency response, making sure you follow all the same instructions as in HW6 \#2d.
		
		\item Apply the filter to the signal $x$, defined above.
		
		\item Plot the Fourier transform of the filtered signal using \lstinline|fft| and \lstinline|plot|. Refer to the lecture examples for the proper way to use FFT and obtain the proper scaling (use one of the two scaling methods mentioned).
	\end{enumerate}
	\makebox[\textwidth][c]{
		\begin{tabular}{r|ccccc}
			\hline
			Type & Type & $F_s$ & $A_{pass}$ & $A_{stop}$ & Frequency specs \\
			\hline\hline
			Butterworth & HPF & \multirow{4}{30pt}{100kHz} & 5dB & 50dB & $F_{stop}=10\text{kHz},F_{pass}=20\text{kHz}$\\
			Chebyshev Type I & LPF & & 2dB & 40dB & $F_{stop}=35\text{kHz},F_{pass}=15\text{kHz}$ \\
			Chebyshev Type II & bandstop & & 5dB & 50dB & $F_{stop}=15\text{kHz},35\text{kHz},F_{pass}=5\text{kHz},45\text{kHz}$ \\
			Elliptic & bandpass & & 5dB & 50dB & $F_{stop}=15\text{kHz},35\text{kHz},F_{pass}=20\text{kHz},30\text{kHz}$ \\
			\hline\hline
		\end{tabular}
	}
\end{document}