\documentclass{article}

\usepackage{amsfonts}
\usepackage{amsmath}
\usepackage{xcolor}
\usepackage{textcomp}
\usepackage{graphicx}
\usepackage{hyperref}
\usepackage{multirow}
\usepackage{minted}

\title{ECE-210-A HW7}
\author{Instructor: Jonathan Lam}
\date{Spring 2022}

\newcommand{\fs}{F_s}

\begin{document}
\maketitle

\makebox[\textwidth][c]{
  \begin{tabular}{r|ccccc}
    \hline\hline
    Type & Type & $A_{pass}$ & $A_{stop}$ & Frequency specification \\
    \hline
    Butterworth & HPF & 5dB & 50dB & $F_{stop}=\fs/10,F_{pass}=\fs/5$\\
    Chebyshev Type I & LPF & 2dB & 40dB & $F_{stop}=\fs/2,F_{pass}=\fs/4$ \\
    Chebyshev Type II & bandstop & 5dB & 50dB & $F_{stop}=\fs/6,fs/3,F_{pass}=\fs/12,5\fs/12$ \\
    Elliptic & bandpass & 5dB & 50dB & $F_{stop}=\fs/12,5\fs/12,F_{pass}=\fs/6,\fs/3$ \\
    \hline\hline
  \end{tabular}
}
\\

\noindent For each of the above scenarios, you will generate a filter, create magnitude-phase plots for the filter, and apply the filter.

To begin with, generate a test signal $x$. This signal will be white noise (sample from a uniform distribution) sampled at $F_s=100\text{kHz}$ over an interval of 2 seconds. If you feel creative, you may choose an alternate signal; be sure to choose a signal with a large frequency content, and specify the sampling rate $F_s$ in a comment.

For each filter, follow these steps. As always, use functions to avoid repeating yourself, and use subplots to organize plots.
\begin{enumerate}
\item Use the given specifications to produce the lowest-order filter which meets the specs. Either:
  \begin{enumerate}
  \item Use \mintinline{matlab}|filterDesigner| to generate a MATLAB function that returns a filter, and then call the function to create the filter object; or
  \item Use the functions for designing and estimating the order of specific types of filters (e.g., \mintinline{matlab}|cheby2ord|, \mintinline{matlab}|cheby2|) to generate it without \mintinline{matlab}|filterDesigner|. Make sure all the parameters are correctly specified! Refer to the example from lecture.
  \end{enumerate}
  
\item Apply \mintinline{matlab}|freqz| on the filter object to produce a frequency-response plot. Similarly to in HW6, don't use this to plot the frequency response: manually plot the frequency response, making sure you follow all the same instructions as in HW6 \#2d.
  
\item Apply the filter to the signal $x$, defined above.
  
\item Plot the Fourier transform of the filtered signal using \mintinline{matlab}|fft| and \mintinline{matlab}|plot|. Refer to the lecture examples for the proper way to use FFT and obtain the proper scaling (use one of the two scaling methods mentioned).
\end{enumerate}
\end{document}

%%% Local Variables:
%%% mode: latex
%%% TeX-master: t
%%% End:
