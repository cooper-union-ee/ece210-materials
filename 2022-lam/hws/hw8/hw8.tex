\documentclass{article}

\usepackage{amsfonts}
\usepackage{amsmath}
\usepackage{xcolor}
\usepackage{textcomp}
\usepackage{graphicx}
\usepackage{hyperref}
\usepackage{multirow}

\usepackage{minted}

\title{ECE-210-A HW8}
\author{Instructor: Jonathan Lam}
\date{Spring 2022}

\begin{document}
\maketitle

\noindent In this problem set you'll be empirically simulating some probability experiments and comparing the results to the theoretical values.

Try your best to avoid loops when possible; vectorize and use logical indexing as much as you can to keep your code neat. There are many ways to do each problem -- have fun with this!

\begin{enumerate}
\item Sample from the Poisson distribution with $\lambda=5$. For each sample, sample
  again from the binomial distribution with $p=0.4$ and n equal to the Poisson
  sample. If the Poisson sample was equal to 0, don’t sample from the binomial
  and just return 0 for that sample. In subplots, plot histograms of the p.m.f.
  from the result of this simulation and a different sampling from the Poisson
  distribution with $\lambda'=\lambda p$.
  
  (If you have background in probability, think of the
  example where people enter a room at rate $\lambda$ and each person is given a prize
  with probability 0.4. Then this distribution is the number of prize
  winners per second.)
  
\item You're going to simulate a problem that becomes relevant in cryptographic hashing algorithms (see the ``birthday attack''). Simulate the following with a few values of $n$, and an appropriate sample size (what is a ``sample'' here?). Report your results clearly using plots or a table (see \mintinline{matlab}|help table|) as appropriate.
  
  Given a classroom of $n$ students with birthdays uniformly distributed throughout the year (and assume no leap years/days):
  
  \begin{enumerate}
  \item What is the probability that at least two students share a birthday? Empirically find the value of $n$ for which the probability is roughly 0.5. (The ``birthday problem.'')
    
  \item Given an arbitrary day of the year $d$, what is the probability that at least one student's birthday falls on that day?
  \end{enumerate}
  
  Do these match the theoretical results? (You can derive or look up formulas to answer these questions.)
  
  \clearpage
\item Go to \url{https://www.varsitytutors.com/gre_math-help/data-analysis/probability}. Simulate answers to at least 4 questions. Make sure to clearly label which questions you are answering, and include comments and/or plots as necessary to explain your thinking.
\end{enumerate}
\end{document}
%%% Local Variables:
%%% mode: latex
%%% TeX-master: t
%%% End:
