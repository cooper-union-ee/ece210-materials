\documentclass{article}

\usepackage{amsfonts}
\usepackage{amsmath}
\usepackage{xcolor}
\usepackage{textcomp}
\usepackage{minted}

\title{ECE-210-A HW2}
\author{Instructor: Jonathan Lam}
\date{Spring 2022}

\begin{document}
\maketitle

\noindent This homework will review numerical integration and differentiation using vectorized operations, some plotting operations. Remember the style guidelines from the previous assignment. For each question, either save the result to a variable or print out the result to the screen (don't suppress the result).

\begin{enumerate}
\item
  Now it's your turn to try out numerical integration and differentiation.
  \begin{enumerate}
  \item Create two vectors of length 100 and 1000, each containing evenly spaced samples of the function $g(t)=\arctan t$ for $0\le t\le 2\pi$.
    
  \item Approximate the derivatives of the vectors using the difference quotient method (i.e., use \mintinline{matlab}|diff| to take the difference between adjacent verbs and divide elementwise by $\Delta t$). Call this estimate $\hat{g}'(t)$.
    
  \item Evaluate the analytical derivative $g'(t)$ over the same intervals of $t$.
    
  \item Say we define the error of the estimate of the derivative to be a normalized mean-square error:
    \begin{equation*}
      \epsilon(g',\hat{g}')=\frac{1}{N}\sum_{t} (g'(t)-\hat{g}'(t))^2
    \end{equation*}
    where $N$ is the number of samples. Calculate the error for both estimates for the derivative -- which is smaller?

    (Note: truncate/pad vectors as necessary.)
    
  \item Approximate the integrals of your original vectors using \mintinline{matlab}|cumtrapz| and \mintinline{matlab}|cumsum|. This will give you four approximations of the integral of $g$. Repeat the steps in parts (c) and (d): evaluate the analytical antiderivative at the same $t$ points, and calculate the error estimates for each of the four estimates of the integral.

    (Note: The antiderivative is $t \arctan\left(t\right)-\frac{1}{2}\ln\left(1+t^{2}\right)+C$.)
    
  \item Plot the best estimate for the integral. Title your plot.
    
  \item \textit{(Optional)} Explore the plotting API: Give the horizontal and vertical axes a label. Turn the grid on/off. Change the axis ticks. Subplots! We will explore more plotting functions in class soon.
    
  \item \textit{(Optional)} Integrate using Simpson's rule and compare results.
  \end{enumerate}
  
  \clearpage
\item Perform the following matrix operations (without \mintinline{matlab}|for| loops). Save each result to a separate variable (i.e., don't alter $A$ after creating it).
  \begin{enumerate}
  \item Generate the matrix:
    \begin{equation*}
      A=\begin{bmatrix}
        10^{0} & 10^{1} & \dots & 10^{4} \\
        10^{5} & 10^6 & \dots & 10^9 \\
        \vdots & \vdots & \ddots & \vdots \\
        10^{45} & 10^{46} & \dots & 10^{49}
      \end{bmatrix}\in M_{10\times 5}(\mathbb{R})
    \end{equation*}
    (Hint: use \mintinline{matlab}|logspace| and \mintinline{matlab}|reshape|.)
    
  \item Flip the third row of $A$ left to right.
    
  \item Create a column vector of the geometric means of each row. (Recall that the geometric mean of $x_1,x_2,\dots,x_n$ is $\sqrt[n]{x_1x_2\dots x_n}$. The \mintinline{matlab}|prod| function will probably be helpful.)
    
  \item Create the submatrix $B\in M_{3\times 3}(\mathbb{R})$ such that $b_{ij}=a_{(i+5)(j+1)}$.
    
  \item Delete rows 5-10 of $A$. (Do this in at most five operations.)
  \end{enumerate}

\item Create a matrix $C$ such that
  \begin{equation*}
    C_{ab}=\frac{\exp(j(a+bj))-\exp(-j(a+bj))}{2j}
  \end{equation*}
  where $-2\pi\le a,b\le 2\pi$ and $a$ and $b$ increment by $10^{-3}$. Generate this matrix in the following ways, and time each method using \mintinline{matlab}|tic|/\mintinline{matlab}|toc|. Write a comment describing the behavior your observations.
  \begin{enumerate}
  \item Using \mintinline{matlab}|for| loops and no pre-allocation.
  \item Using \mintinline{matlab}|for| loops and pre-allocation.
  \item Repeat parts (a) and (b), but flip the order of your loops.
  \item Using \mintinline{matlab}|meshgrid|.
  \item (Optional) Using broadcasting.
  \end{enumerate}

  (Note: make sure to properly clear variables to make sure you see the effects of pre-allocation.)
    
\item \textit{(Optional)} Try these problems using Python and numpy.
  
\item \textit{(Optional)} How does vectorization work (in MATLAB or in general)? Do a little research and write a few sentences explaining in your own words how it achieves speedup over \mintinline{matlab}|for| loops.
\end{enumerate}

\end{document}
%%% Local Variables:
%%% mode: latex
%%% TeX-master: t
%%% End:
